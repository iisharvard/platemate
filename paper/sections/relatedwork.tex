\section{RELATED WORK}

%Prior work makes clear that eating depends on a variety of
%environmental and personal factors, with accurate nutritional
%information as a ``necessary but not sufficient'' condition for
%positive eating behavior
%modification~\cite{worsley2002nutrition}. Recording consumption is
%useful for dieting in particular~\cite{baker1993self}, but nutrition
%knowledge is also correlated with healthy eating more
%broadly~\cite{wardle2000nutrition}.


Nutritionists have established several methods for measuring food
intake. One prominent approach is 24-hour recall, in which a trained
dietitian interviews a subject about her consumption over the previous
day~\cite{martin2009novel}. Accuracy depends on the subject's memory
and honesty, and the technique requires a costly expert to conduct
analysis. The main alternative is food journals, in which subjects
record meals and estimate portions themselves, usually with pen and
paper.

Both methods require significant time and self-reports also suffer from limited
accuracy. A review of nine studies found error rates from $-76\%$
(underestimates) to $+24\%$
(overestimates)~\cite{schoeller1990inaccuracies}. Prior work also
suggests a dangerous bias in self-report methods. Most subjects
selectively underreport fat intake, and obese people underestimate
portions much more than leaner
ones~\cite{pikholz2004under,goris2000undereating}. These errors imply
a larger problem of self-deception, especially in vulnerable groups.

A number of online interfaces exist to simplify the process of food
logging. Smartphone applications and online calorie databases improve
on earlier methods by performing calculations automatically. However,
they still require tedious logging that discourages
recording. Self-reports using these interfaces are no more accurate
than pen and paper~\cite{ann2006use,yon2007personal}.

The Computer Science community has explored additional alternatives,
such as automatic analysis of chewing sounds~\cite{amft05:analysis}
and scanned grocery receipts~\cite{mankoff02:using}.  These methods,
while potentially more scalable and less time-consuming than current
approaches, remain inaccurate.

Martin et al. recently suggested an alternative approach called the
Remote Food Photography Method (RFPM)~\cite{martin2009novel}. Rather
than typing names of foods and estimating portions, users take photographs of their plates \newcontent{both at the beginning
  of the meal and at the end to accurately capture how much food was actually eaten.} Trained dietitians identify the pictured foods remotely
and estimate portions.  The results of laboratory studies showed that
dietitians using RFPM underestimated calories by
5-7\% compared to the ground truth obtained by directly weighing the foods~\cite{martin2009novel}.

RFPM thus combines the accuracy of direct observation by experts with the
convenience of free-living conditions. Users of the method found it
extremely satisfying and easy to use~\cite{martin2009novel}. The
problem is cost. RFPM relies on experts to analyze each
photograph, limiting the system's accessibility and potential
scale. %The method might be feasible in specific healthcare settings, but trained dietitians are too scarce for general use.

Kitamura et al. attempted to use computer vision to cheaply implement
RFPM~\cite{kitamura2010image}.  They were successful in
algorithmically detecting if a photograph contained food and in
estimating amounts of general categories of food, such as meats,
grains, and fruit.  They did not attempt to identify the specific
foods in a photo or provide actual intake totals.

The cost of experts and limitations of computer vision suggest an
opportunity for crowdsourced nutritional analysis. Prior research indicates that
the most difficult part of nutritional analysis is estimating portion
size~\cite{martin2009novel}, and that trained amateurs have low bias
but high variance~\cite{martin2007empirical}. The ``wisdom of crowds''
is ideally suited to these situations, since the average of amateur
estimates often beats a single expert~\cite{wisdom}. 
%Crowdsourcing
%using today's tools, however, creates new challenges. Workers are
%untrained, and their incentives are not the same as those of
%professional nutritionists working with a specific patient.

A recent iPhone application demonstrates, however, that naive approaches to
crowdsourcing for nutritional analysis are not sufficient. In April, 2011, the fitness
website Daily Burn released Meal Snap, which allows users to
photograph foods and receive calorie estimates by so-called ``pure
magic.''\footnote{\url{http://mealsnap.com/}, accessed July 5, 2011} Meal Snap creates
\newcontent{a single Mechanical Turk task for each image.} Workers
provide a free text description of food, and the application appears
to match this description with a database of average consumption to
estimate a range of possible calories. This approach is appealing, but
critics have accused it of failing to provide useful
data\footnote{\url{http://www.mobilecrunch.com/2011/04/05/too-lazy-to-count-calories-now-you-can-}\\\url{just-take-a-picture-of-your-meal/}} and our evaluation showed that Meal Snap's results do not correlate with the meal's actual caloric content.


\newcontent{PhotoCalorie\footnote{\url{http://photocalorie.com/},
    accessed on July 5, 2011} is a recent on-line tool that encourages
  users to upload photographs of their meals, but it uses them just to
  illustrate the user's personal photo journal.  The apparent similarity to PlateMate is superficial because to obtain calorie
  estimates, users have to enter short descriptions of the contents of
  the meals and manually estimate the amounts eaten.}
